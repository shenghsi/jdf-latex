\input{ main-style }

\title{
  {{-cookiecutter.project_title_line_1-}} \\

  
    {{-cookiecutter.project_title_line_2-}} \\
  

  
    {{-cookiecutter.project_title_line_3-}} \\
  
  }

\authoremail{ {{-cookiecutter.author_name-}} }{ {{-cookiecutter.author_email-}} }

\begin{document}
\maketitle
\thispagestyle{fancy}

% You may include a new tex file here using:
% \input{<file_name>.tex}

\begin{abstract}
If your paper requires an abstract
\end{abstract}

\section{Presentational elements}
\begin{figure}[H]
  \centering
  \includegraphics[scale=0.7]{figs/example-image1}
  \caption{Palatino. Make sure the live text (top) uses the same font as the image (bottom).}
  \label{fig::1}
\end{figure}

Figures~\ref{fig::1}  

\begin{itemize}
\item
  Here's an item.
\item
  at the same level of indentation.
\end{itemize}

Bulleted lists follow the same format:
%
\begin{enumerate}
\item
  This is an item.
\end{enumerate}

\notext
\nocite{*}
\bibliographystyle{apacite}
\bibliography{references}

\clearpage
\end{document}
